\documentclass[hyperref={pdfpagelabels=true}]{beamer}
\usepackage{lmodern}
\usepackage{graphicx}
\usepackage{tikz}
\usepackage[framemethod=tikz]{mdframed}
\usepackage{ragged2e}

%%%%%%%%%%%%%%%%%%%%%%%%%%%%%%%%%%%%%%%%%%%%%%%%%%%%%%%%%%%%%%%%%%%%%%%%%%%%%%%%%%%%%%%%%%%%%%%%%
%This work is licensed under a Creative Commons Attribution-ShareAlike 4.0 International License.
%
%You are free to:
%
%    Share — copy and redistribute the material in any medium or format
%    Adapt — remix, transform, and build upon the material
%    for any purpose, even commercially.
%
%    The licensor cannot revoke these freedoms as long as you follow the license terms.
%
%Attribution — You must give appropriate credit, provide a link to the license, and indicate if changes were made. You may do so in any reasonable manner, but not in any way that suggests the licensor endorses you or your use.
%
%ShareAlike — If you remix, transform, or build upon the material, you must distribute your contributions under the same license as the original. 
%
%%%%%%%%%%%%%%%%%%%%%%%%%%%%%%%%%%%%%%%%%%%%%%%%%%%%%%%%%%%%%%%%%%%%%%%%%%%%%%%%%%%%%%%%%%%%%%%%%

\newmdenv[tikzsetting={draw=black,fill=white,fill opacity=0.5, line width=0pt},
      backgroundcolor=none,leftmargin=5,rightmargin=5,innertopmargin=4pt]
      {titleBox}


\definecolor{dred}{rgb}{0.647059, 0.164706, 0.164706}
\definecolor{dgreen}{rgb}{0., 0.545098, 0.545098}
%\usecolortheme[named=dred]{structure}

%\usetheme{Marburg}
%\usecolortheme{crane}

%\usetheme{AnnArbor}
%\usecolortheme{wolverine}

\usetheme{Warsaw}
\usecolortheme{dolphin}

%\title{Horizontes no Panorama da Ci\^{e}ncia de Dados Espaciais}
\title{Ci\^{e}ncia de Dados Espaciais}
\subtitle{Aonde Vamos?}
\author{Joana Sim\~{o}es} 

\author[shortname]{Joana Sim\~{o}es \inst{1}}
\institute[shortinst]{\inst{1} Eurecat, Centro Tecnol\'{o}gico da Catalunha}

%\date{\today} 
%\titlegraphic{\includegraphics[width=.35\textwidth]{3d2.png}}

\usepackage{listings}

\newcommand{\soooo}{H$_2$SO$_4$}

%fdl stuff
\usepackage{hyperref}
\hypersetup{colorlinks, 
           citecolor=black,
           filecolor=black,
           linkcolor=black,
           urlcolor=black,
           bookmarksopen=true,
           pdftex}

\hfuzz = .6pt % avoid black boxes

\lstset{language=SQL}



\begin{document}
\setbeamertemplate{footline}[page number]
\setbeamertemplate{navigation symbols}{}

%\titlepage

{ \usebackgroundtemplate{\includegraphics[width=\paperwidth]{3d2.png}} 
\begin{frame}[plain]
    \vspace{10em}
    \begin{titleBox}
        \centering \textbf{Ci\^{e}ncia de Dados Espaciais}\\
        \textbf{Aonde Vamos?}\\  
        \justify
        \small{Joana Sim\~{o}es}\\
        \small{Eurecat, Centro Tecnol\'{o}gico da Catalunha\\
        }
        
    \end{titleBox}

%\frametitle{Frame with nice background} 
%\begin{itemize} \item 1 \item 2 \item 3 \end{itemize}
\end{frame} 
} 
 
\begin{frame}
\frametitle{Tabela de Conte\'{u}dos}
\tiny{
\tableofcontents}
\end{frame}

%TODO: change layout, atribuitions, trocar cont. por exemplo

\section{Introdu\c{c}\~{a}o} 
\begin{frame}
\frametitle{Cientistas \& Unic\'{o}rnios}
\small{ 
      \begin{itemize}    
        \item<1->``Data Scientist'' is a Data Analyst who lives in California.
        \item<1->A data scientist is someone \textit{who is better at statistics than any software engineer and better at software engineering than any statistician.} (Wills, Cloudera)
      \end{itemize}                
}
    \begin{figure}   
         \includegraphics[width=0.5\textwidth]{venn.png}   
    \end{figure}     


\end{frame}


\begin{frame}
\frametitle{Dados, Informa\c{c}\~{a}o e Conhecimento}

\small{ 
      \begin{itemize}    
        \item<1->Dados sao os factos que descrevem o mundo.%e.g.: temperatura, idade, numero de degraus da escada de casa
        \item<1->A informacao surge quando transformamos esses valores em algo relevante.%Pode ajudar-nos a tomar decisoes informadas
        \item<1->Conhecimento implica uma generalizacao dos dados e da informacao, de forma a criar regras.%'Nteste processo podemos simular a inteligencia humana, atraves de modelos analiticos
      \end{itemize}                
}

    \begin{figure}   
         \includegraphics[width=0.5\textwidth]{data.png}   
    \end{figure}     


\end{frame}

\begin{frame}
\frametitle{Algo \textit{espacial}...}

Lei de Tobler:\\
\textit{Everything is related to everything else, but near things are more related to each other.}%First Law of Geography
%behind buffers or heatmaps

\begin{columns}
  \begin{column}{0.4\textwidth}
    \begin{figure}  
	\includegraphics[width=\textwidth]{cruise.png}\\
           \tiny{Heatmap de Tweets perto de um Cruzeiro.}%higher epsilon and lower minpts  	          
       \end{figure}             
  \end{column}
  \begin{column}{0.6\textwidth}
      \begin{figure}  
	\includegraphics[width=\textwidth]{schools.png}\\
           \tiny{Buffers de acidentes ao redor de escolas.}%lower epsilon and higher minpts  	          
       \end{figure}  
  \end{column}  
\end{columns}

\end{frame}

\begin{frame}
\frametitle{Problemas Fundamentais}

Longley \textit{ET AL} (2005):
\small{ 
      \begin{itemize} 
      \item<2->O comportamento espacial actual, muitas vezes reflecte padroes passados.%necessidade de estudar series temporais
        \item<3->A explicacao no tempo apenas necessita de olhar para o passado, mas a explicacao no espaco necessita de olhar em todas as direccoes simultaneamente.%A autocorrelacao temporal tem uma dimensao, enquanto que a autocorrelacao espacial tem duas ou tres dimensoes
        \item<4->Embora alguns fenomenos espaciais variem de forma gradual atraves do espaco, outros podem exibir uma extrema irregularidade.%break Toblers Law. Por exemplo acidentes de trafico
        \item<5->Embora a autocorrelacao espacial nos ajude a construir representacoes, ela pode frustrar os nossos esforcos de predicao.% patterns of spatial autocorrelation in one variable, would be likely mirrored on others, but that does not imply any causality.
      \end{itemize}                
}

\end{frame}


\section{Tend\^{e}ncias} 
\begin{frame}
\frametitle{Onde Vamos?}

    \begin{figure}   
         \includegraphics[width=0.7\textwidth]{fortune-teller.png}   
    \end{figure}     

\end{frame}


\begin{frame}
\frametitle{(Algumas) Tend\^{e}ncias}

    \begin{figure}   
         \includegraphics[width=\textwidth]{trends.png}   
    \end{figure}     

\end{frame}

\begin{frame}
\frametitle{Import\^{a}ncia Crescente}

      \begin{itemize}    %ubiquidade da ciencia de dados, e nao so reservada a investigacao
        \item<1->A ciencia de dados ganha torna-se cada vez mais importante, a medida que entra em novos campos.%personal health eHealth
        \item<2->A Geografia tambem benefia desse impulso, a medida que o publico em geral toma consciencia de que a maior parte das coisas, acontecem num lugar.
      \end{itemize}                
      
    \begin{figure}   
         \includegraphics[width=0.4\textwidth]{runkeeper.jpg}   
    \end{figure} 

\end{frame}

\begin{frame}
\frametitle{Import\^{a}ncia Crescente (cont.)}

      \begin{itemize}
        \item<1->\textit{Whitings} e uma balanca ``inteligente''.
        \item<2->Recolhe medicoes corporais precisas: peso, massa gorda e batimentos cardiacos.
        \item<3->Uma app analisa estes dados, mostra tendencias e permite gerar planos e monitorizar metas.%por exemplo para perder peso 
      \end{itemize}                
      
\begin{columns}
  \begin{column}{0.7\textwidth}
    \begin{figure}  
	\includegraphics[width=0.7\textwidth]{scale.png}\\
       \end{figure}             
  \end{column}
  \begin{column}{0.3\textwidth}
      \begin{figure}  
	\includegraphics[width=0.7\textwidth]{app.png}\\
       \end{figure}  
  \end{column}  
\end{columns}
\end{frame}

%Outro exemplo: seguranca pessoal de idosos?



\begin{frame}
\frametitle{Mais Dados}
%Automaticos
      \begin{itemize}
        \item<1->Existe uma explosao na quantidade de dados gerados por sensores (IoT, IoE).
        \item<2->Gracas a tecnologias de posicionamento mais baratas e generalizadas (e.g.: receptores de GPS), uma proporcao grande destes dados esta georeferenciada.
      \end{itemize}                
      
\begin{columns}
  \begin{column}{0.5\textwidth}
    \begin{figure}  
	\includegraphics[width=0.7\textwidth]{cisco.png}\\
       \end{figure}             
  \end{column}
  \begin{column}{0.5\textwidth}
      \begin{figure}  
	\includegraphics[width=0.7\textwidth]{drones.png}\\
       \end{figure}  
  \end{column}  
\end{columns}

\end{frame}

\begin{frame}
\frametitle{Mais Dados (cont.)}
%Automaticos
      \begin{itemize}
        \item<1->LIDAR: mede propriedades da luz reflectida de modo a obter a distancia ou outra informacao a respeito um determinado objecto distante.
        \item<2->O UAV LIDAR aplica esta tecnica a partir de um drone.%permite fazer levantamentos geodesicos muito acurados, e com baixo custo; funciona em sitios com nuvens, ou em tuneis, onde o sinal de GPS 'e limitado
        %TODO: arranjar um exemplo
        %e airborne ops run for 7 days only and acquired billions of laser points and thousands of air photo on project area blocks encompassing approximately 29,000 Ha - See more at: http://blog.lidarnews.com/uavs-set-to-revolutionize-archaeological-mapping/#sthash.0aofVAFp.dpuf
      \end{itemize}                
%tempos excitantes para a arqueolologia
    \begin{figure}   
         \includegraphics[width=0.8\textwidth]{lidar.jpg} %choque com um aviao  
    \end{figure} 
            
\end{frame}

\begin{frame}
\frametitle{E Ainda Mais Dados}
%Nao-automaticos
      \begin{itemize}
        \item<1->Existe uma explosao na quantidade de UGC.
        \item<1->voluntario: citizens as sensors e VGI (p.e: OSM).
        \item<1->Nao-voluntario (?): gerados nas redes sociais (p.e: Twitter).        
      \end{itemize}                

    \begin{columns}
    \begin{column}{0.5\textwidth}
        \begin{figure}  
            \includegraphics[width=0.7\textwidth]{osm_members.png}\\
            \tiny{Evolucao do numero de membros do OSM.}
        \end{figure}             
    \end{column}
    \begin{column}{0.5\textwidth}
        \begin{figure}  
            \includegraphics[width=0.7\textwidth]{twittermap.png}\\
            \tiny{Mapa com a localizacao do ultimo milhao de Tweets.}
        \end{figure}  
    \end{column}  
    \end{columns} 

\end{frame}


\begin{frame}
\frametitle{E Ainda Mais Dados (cont.)}

      \begin{itemize}
        %\item<1->Com a chegada de cruzeiros ao Porto de Barcelona, ha uma explosao no numero de Tweets.
        \item<2->Um metodo de deteccao de origem, permite diferenciar os Tweets gerados por locais dos gerados por estrangeiros.
        \item<3->Podemos observar que ha um acrescimo no numero de Tweets gerados por estrangeiros, apos a chegada dos cruzeiros ao Porto de Barcelona.
        %\item<1->Com base nesta classificacao, podemos observar que estes dois grupos se distribuem diferentemente pela cidade.        
      \end{itemize}                

    \begin{columns}
    \begin{column}{0.5\textwidth}
        \begin{figure}  
            \includegraphics[width=0.7\textwidth]{normal_vs_foreign_CapSetmanaCreuers.png}\\
            \tiny{Clusters de Tweets de locais e estrangeiros.}
        \end{figure}             
    \end{column}
    \begin{column}{0.5\textwidth}
        \begin{figure}  
            \includegraphics[width=0.7\textwidth]{barriosForeignUsersCapSetmanaCreuers_cnt.png}\\
            \tiny{Distribuicao de densidades de Tweets de estrangeiros.}
        \end{figure}  
    \end{column}  
    \end{columns} 

\end{frame}


\begin{frame}
\frametitle{``Arqueologia'' de Dados}

      \begin{itemize}
        \item<1->Existe uma necessidade crescente de ``recuperar'' fontes de dados de legacy (por exemplo em formato analogico)
        \item<2->Embora isto implique um desafio tecnologico, muitas vezes estes conjuntos de dados sao extremamente valiosos.%explicar porque
      \end{itemize}                
    \begin{figure}   
         \includegraphics[width=0.8\textwidth]{snow.jpg}\\
         \tiny{Mapa de casos de colera em Londres, produzido por John Snow (1864).}
    \end{figure} 

\end{frame}

\begin{frame}
\frametitle{``Arqueologia'' de Dados (cont.)}

      \begin{itemize}      
        \item<1->Dados de informacao de pescas (biomassas), foram guardados sem metadados.
        \item<2->Para capturar estes dados, foi necessario efectuar cruzeiros com um custo elevadissimo.%valor em $$$
        \item<3->A informacao capturada e irrepetivel.%valor em unicidade        
        \item<4->Os metadados estao encodificados na path do ficheiro (i.e.: timestamp, nome do cruzeiro).%foi necessario criar uma aplicacao que navegue e interprete as paths no disco, e as associe aos ficheiros
        %A aplicacao tem de ser monitorizada por um operador
      \end{itemize}                
    \begin{figure}   
         \includegraphics[width=0.8\textwidth]{fao.png}
    \end{figure} 
\end{frame}

\begin{frame}
\frametitle{Tecnologias de \textit{Big Data}}

      \begin{itemize}      
        \item<1->3 Vs.
        \item<2->Computacao em ambiente Cloud, NoSQL, Processamento em tempo-real ,Linked Data.%web semantica
        \item<4->Big Spatial Data.
      \end{itemize}                
    \begin{figure}   
         \includegraphics[width=0.5\textwidth]{3vs.png}
    \end{figure} 

\end{frame}


\frametitle{Tecnologias de \textit{Big Data} (cont.)}
\begin{frame}
Desenvolvimento de tecnologias espaciais escalaveis e distribuidas.
      \begin{itemize}      
        \item<2->Suporte espacial ainda limitado.%implementado em poucas tecnologias e com limitacoes
        \item<3->Necessidade crescente de saber em que situacoes utilizar cada tecnologia.%trade offs nem sempre sao claros: benchmarking
      \end{itemize}                
    \begin{figure}   
         \includegraphics[width=0.7\textwidth]{general1.png}\\
         \tiny{Benchmarking de bases de dados espaciais na cloud;.}%relacionais vs cluster         
    \end{figure} 
\end{frame}

\begin{frame}
\frametitle{Uso Cada vez Mais Generalizado de \textit{ML}}
A disponibilidade crescente de dados e os progressos computacionais vao impulsionar o uso de Machine Learning.
      \begin{itemize}      
        \item<2->Ascendencia algoritmos computacionalmente intensivos (p.e.: Deep Learning)%
        \item<3->Modelos Ensemble.%Grid search for parameters
        \item<4->Modelos espaciais.%Mts dimensoes
        \item<5->Descriptivo $\rightarrow$ Predictivo
      \end{itemize}                
    \begin{figure}   
         \includegraphics[width=0.8\textwidth]{deep.png}\\
         \tiny{Arquitectura de uma red de Deep Learning para reconhecimento facial.}
    \end{figure} 
\end{frame}

\begin{frame}
\frametitle{Uso Cada vez Mais Generalizado de \textit{ML} (cont.)}

    Trafico e um dominio onde tradicionalmente se utilizam modelos de micro-simulacao.
      \begin{itemize}         
        \item<2->Foram aplicados modelos de ML para prever os tempos de viagem na cidade de Barcelona.
        \item<3->Os modelos foram treinados e validados com um ano de dados.
        \item<4->A ANN foi descartada a favor de uma SVM.
       \end{itemize}            
          
        \begin{figure}   
         \includegraphics[width=0.6\textwidth]{prediction.png}\\
         \tiny{Ajuste entre as previsoes do SVM (verde) e os valores observados (vermelho).}
    \end{figure} 


\end{frame}


\begin{frame}
\frametitle{Jornalismo de Dados \& \textit{Story Telling}}

%TODO: explain story telling; exemplo Kibana de acidentes
% visualizacao: em outro slide?

\end{frame}

\begin{frame}
\frametitle{Mais e Melhores Dados Abertos}

%TODO: explain Open Data; falar do OpenGOV

\end{frame}

\begin{frame}
\frametitle{\'{E}tica}

%TODO: spatial obfuscation

\end{frame}

\section{Consideracoes Finais} 

\begin{frame}
\frametitle{Import\^{a}ncia destas Tend\^{e}ncias para o FOSS4G}

%TODO: SIG Open-Source
%Free and Open Source Software for Geospatia

%ha mais dados e mais pessoas a quererem trabalhar com eles...

%Areas onde o FOSS4Geo se pode import

%standards de dados abertos (formatos, servicos); nao existem formatos standard para IoT
%bibliotecas de machine learning com capacidades espaciais (QGIS, R)
%visualiacao: D3
%infra estrutura e processamento de Big Spatial Data

\end{frame}

%Slide final

%importancia do software aberto para a Etica> Trust
%Nao ha backdoors ou codigo mailicioso


\begin{frame}
\frametitle{Obrigada pela vossa Aten\c{c}\~{a}o}
    \begin{figure}   
      \includegraphics[width=0.6\textwidth]{end.jpg}      
    \end{figure}   
    Esta apresenta\c{c}\~{a}o encontra-se dispon\'{i}vel em: \centering{\\ \url{http://tinyurl.com/nfbrhvl}\\}
      \vspace{5mm}    
\end{frame}


\end{document}


